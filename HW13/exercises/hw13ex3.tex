% chktex-file 1 chktex-file 3 chktex-file 9 chktex-file 17 chktex-file 18 chktex-file 36

\section{Exercise 3.}

Let \(U \subset \mathbb{C}\) be open and let \(g_n : U \to \mathbb{C}\) be holomorphic functions such that \(\sum_{n=1}^{\infty} |g_n|\) converges compactly in \(U\). Define
\[
f_n(x) = \prod_{j=1}^{n} (1 + g_j(x)).
\]
(a) Show that \((f_n)_{n \in \mathbb{N}}\) converges compactly to a holomorphic function \(f : U \to \mathbb{C}\).

(b) Let \(z_0 \in U\). Show that \(f(z_0) = 0\) if and only if there exists \(j \in \mathbb{N}\) such that \(g_j(z_0) = -1\), that there are finitely many such \(j\), and that the order of the zero \(z_0\) for \(f\) is equal to the sum of the multiplicities of \(z_0\) as a zero of all the functions \(1 + g_j\).

\subsection*{Solution Part (a)}

Let $K \subseteq U$ be a compact set. Then, $\sum_{n = 1}^{\infty}|g_n|$ converges uniformly in $K$, so apply exercise 2 to conclude that $f_n$ converges uniformly to a function $f_K$ in $K$. By Weierstrass theorem, $f_K$ must be holomorphic.

Now for any $z \in U$ let $K \subset U$ be a compact set that contains $z$. Define $f(z) = f_{K_z}(z)$. $f$ is well defined because if we take another compact set $K_z'$ that contains $z$, then by uniqueness of limit, $f_{K_z}(z) = f_{K_z\cap K_z'}(z) = f_{K_z'}(z)$.

Fix $z \in U$. We want to show that there exists a neighborhood of $z$ for which $f$ is holomorphic. Since the choosing of the compact set $K_z$ doesn't affect the value of $f(z)$, let $\varepsilon > 0$ such that $\ol{B_{\varepsilon}(z)} \subset U$ and let $K_w = \ol{B_{\varepsilon }(z)}$ for every $w \in B_{\varepsilon}(z)$. With this choosing of $K_w = K_z$ we get that $f(w) = f_{K_z}(w)$, and since $f_{K_z}$ is holomorphic, it follows that $f$ is holomorphic at $B_\varepsilon(z)$. Thus, $f$ is holomorphic.

Finally, $f_n$ converges compactly to $f$ because for any compact set $K$, $f_n$ converges uniformly to $f_K$, and again, by uniqueness of limit, $f(z) = f_K(z)$ for every $z \in K$.

\subsection*{Solution Part (b)}

If $f(z_0) = 0$ for $z_0 \in U$, then there exists $j \in \N$ for which $1+g_j(z) = 0$, otherwise we would get the same contradiction we formulated at exercise 2(b).

If for some reason there exists an infinite number of $j\in \N$ for which $1+g_j(z_0) = 0$, then for every $n\in \N$ the tail $\prod_{j = n+1}^{\infty} (1+g_j(z_0)) = 0 \not\to 1$ (doesn't converge to 1), so the product doesn't converge to $f$ according to the definition.


Now, let $J = \{j_1,\ldots, j_q\}$ be the set of indices for which $g_{j_i}(z_0) = -1$. Then, for each of this indices, we can factorize the zeros of order say $m_i$ to obtain $1+g_{j_i}(z) = (z-z_0)^{m_i}\cdot h_i(z)$ for some holomorphic function that doesn't vanish at $z_0$. Then,
\[ \everymath{\displaystyle}
\arraycolsep=1.8pt\def\arraystretch{2.5}
\begin{array}{rcl}
    f(z) & = & \prod_{j\in J} (1+g_j(z)) \times \prod_{j\not\in J} (1+g_j(z))\\
    & = & (z-z_0)^{m_1+\cdots+m_q}\cdot \prod_{i = 1}^q h_i(z) \times \prod_{j\in \N\backslash J} (1+g_j(z)).
\end{array}\]

From the definition of $h_i$, we know that $\prod_{i = 1}^q h_i(z_0) \neq 0$ and we defined $J$ in such way that there are no indices outside of $J$ for which $1+g_j(z_0) = 0$. Therefore, $f$ has a zero of order $m_1+\cdots+m_q$ at $z_0$.