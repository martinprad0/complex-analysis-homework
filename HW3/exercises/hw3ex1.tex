% chktex-file 3 chktex-file 18 chktex-file 36
\section*{Exercise 1.}

\begin{enumerate}[label=(\alph*)]
    \item Calculate $\displaystyle \oint_{|z-1| = 2} z^{n}\sin(z) dz$ for $n\in\Z$.
    \item For $n \in \N_0$ prove that
    \[ \int_{|z+2i| = 3} \frac{1}{(z^2+\pi^2)^{n+1}}dz = \frac{-(2n)!}{(n!)^2}(2\pi)^{-2n} \]
\end{enumerate}

\subsection*{Solution Part (a)}
When $n\geq 0$, $z \mapsto z^n \sin(z)$ is an entire function with Taylor series
\[ z^n \sin(z) = \sum_{k = 0}^\infty \frac{(-1)^k}{(2k+1)!}x^{2k+1+n}.  \]
Therefore, using Cauchy's theorem, we assert that
\[ \oint_{|z-1| = 2} z^{n}\sin(z) dz = 0.  \]
Finally, for the negative case, let $n \in \Z^+$ and note that by using Cauchy formula we obtain
\[ \int_{|z-1| = 2} \frac{\sin(z)}{z^{n}} dz = \frac{2\pi i}{n!} (\sin)^{(n-1)}(0) = \begin{cases}
    0, & n \equiv 1,3 \mod 4\\
    1, & n \equiv 2 \mod 4\\
    -1,& n \equiv 0 \mod 4.
\end{cases} \]

\subsection*{Solution Part (b)}

\begin{center}
    \includegraphics*[width=0.8\textwidth]{../pictures/hw3ex1pic1.png}
\end{center}

Let $\displaystyle f(z) = \frac{1}{(z-i\pi)^{n+1}} = (z-i\pi)^{-(n+1)}$ for $z \in \C\backslash\{i\pi\}$, and note (from the image above) that $f$ is analytic on the disk $\{z\in\C \;:\; |z+2i| \leq 3\}$. Therefore, we can use Cauchy's formula to conclude that

\[ \frac{f^{(n)}(-i\pi) \cdot 2\pi i}{n!} = \int_{|z+2i| = 3} \frac{f(z)}{(z+i\pi)^{n+1}} dz = \int_{|z+2i| = 3} \frac{1}{(z^2+\pi^2)^{n+1}}dz. \]
Then,
\[ \everymath{\displaystyle}
\arraycolsep=1.8pt\def\arraystretch{1.5}
\begin{array}{rcl}
    f^{(1)}(z) & = & (-(n+1))(z-i\pi)^{-(n+2)}\\
    f^{(2)}(z) & = & (-(n+1))(-(n+2))(z-i\pi)^{-(n+3)}\\
    \vdots & & \vdots\\
    f^{(n)}(z) & = & (-(n+1))\cdots (-2n)(z-i\pi)^{-(2n+1)}\\
    & = & \frac{2n!}{n!}(z-i\pi)^{-(2n+1)}
\end{array} \]
Finally, by putting everything together, we obtain
\[ \everymath{\displaystyle}
\arraycolsep=1.8pt\def\arraystretch{2.5}
\begin{array}{rcl}
    \int_{|z+2i| = 3} \frac{1}{(z^2+\pi^2)^{n+1}}dz & = & \frac{f^{(n)}(-i\pi) \cdot 2\pi i}{n!}\\
    & = & \frac{2n!\cdot (2\pi i)}{(n!)^2\cdot(-2\pi i)^{2n+1}}\\
    & = & \frac{-(2n)!}{(n!)^2}(2\pi)^{-2n} .
\end{array} \]

Also, I made the case $n = 0$ using partial fractions:


\[ \everymath{\displaystyle}
\arraycolsep=1.8pt\def\arraystretch{2.5}
\begin{array}{rcl}
    \int_{|z+2i| = 3} \frac{1}{z^2+\pi^2}dz & = & \int_{|z+2i| = 3} \frac{1}{(z+i\pi)(z-i\pi)}dz\\
    & = & \frac{-1}{2\pi i}\int_{|z+2i| = 3} \frac{1}{z+i\pi} - \frac{1}{z-i\pi}dz\\
    & = & \frac{-1}{2\pi i}\int_{|z+2i| = 3} \frac{1}{z+i\pi} + \frac{1}{2\pi i}\int_{|z+2i| = 3} \frac{1}{z-i\pi}dz\\
    & = & \frac{-1}{2\pi i}\int_{|z+2i| = 3} \frac{1}{z+i\pi} + 0\\
    & = &  \frac{-1}{2\pi i} \cdot 2\pi i\\
    & = & \frac{-(2\cdot 0)!}{(0)!^2} (2\pi)^{2\cdot 0}.
\end{array}  \]


