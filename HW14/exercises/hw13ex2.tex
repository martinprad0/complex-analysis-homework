% chktex-file 3 chktex-file 9 chktex-file 17 chktex-file 18 chktex-file 36

\section{Exercise 2.}

Prove the class theorem:  
Let \((X, d)\) be a compact metric space and let \(g_n : X \to \mathbb{C}\) be continuous functions such that \(\sum_{n=1}^{\infty} |g_n|\) converges uniformly. Define \(f_n : X \to \mathbb{C}\) by
\[
f_n(x) = \prod_{j=1}^{n} (1 + g_j(x)).
\]
We already know that for every \(x \in X\), the product \(\prod_{n=1}^{\infty} (1 + g_j(x))\) is absolutely convergent. Then
\[
f : X \to \mathbb{C}, \quad f(x) := \lim_{n \to \infty} f_n(x)
\]
is well-defined.

Show that (a) \(f_n \to f\) uniformly and (b) that there exists \(N \in \mathbb{N}\) such that for all \(x \in X\),
\[
f(x) = 0 \iff g_n(x) = -1 \text{ for some } n \leq N.
\]

\subsection*{Solution Item (a)}

\textbf{Claim 1:} for any real number $x$, $x+1\leq e^x$.

\textit{Proof:} The function $F(x) = e^x - x - 1$ has derivative $F'(x) = e^x - 1$ which has a critical point at $x = 0$. The function is convex because $F''(x) = e^x > 0$ and thus, $x= 0$ is a global minimum of $F$. Since $F(0) = 0$, it follows that for any $x \in \R$, $F(x) \geq F(0) = 0$, and thus, $e^x - x - 1 \geq 0$.\\.\hfill $\square$

\textbf{Claim 2:} For an absolutely convergent sequence $(a_n)_{n\in \N}$, that is $\sum_{n = 1}^{\infty} |a_n| < \infty$,
\[ \left| \prod_{k = 1}^{n} (1+a_k) - 1\right| \leq \prod_{k = 1}^n (1+ |a_k|) - 1. \]

\textit{Proof:} By the triangle inequality, any polynomial $P(a_1,\ldots, a_n)$ satisfies
\[ |P(a_1,\ldots, a_n)| \leq P(|a_1|,\ldots, |a_n|). \]
Therefore, by taking the polynomial $P_n(a_1,\ldots, a_n) = \prod_{k = 1}^{n} (1+a_k) - 1$, we obtain the desired result.\\.\hfill $\square$

\textbf{Claim 3:} For an absolutely convergent sequence $(a_n)_{n\in \N}$,
\[ \left| \prod_{k = 1}^{\infty} (1+a_k) - 1 \right| \leq \exp\left( \sum_{k = 1}^{\infty} |a_k| \right) - 1. \]

\textit{Proof:} we know that both $\lim_n \prod_{k = 1}^{n} (1+a_k)$ and $\lim_n \sum_{k = 1}^{n}|a_k|$ exist from the hypothesis that $a_n$ is absolutely convergent. Use \textbf{Claim 2} and \textbf{Claim 1} to conclude that
\[ \left| \prod_{k = 1}^{n} (1+a_k) - 1 \right| \leq  \prod_{k = 1}^{n} (1+|a_k|) - 1 \leq \exp\left( \sum_{k = 1}^n |a_k|\right) - 1.\]
Therefore, after taking limits on both sides we obtain the desired result. In fact, this exact same argument also works on the tails:
\[ \left| \prod_{k = N+1}^{\infty} (1+a_k) - 1 \right| \leq \exp\left( \sum_{k = N+1}^{\infty} |a_k| \right) - 1,\quad \forall N \in \N. \]
.\hfill $\square$

Now, note that we can factorize the first $n$ product terms of $f_n$ from $f$ 
\[ f(z) - f_n(z) =  \left( \frac{f(z)}{f_n(z)} - 1 \right)\cdot f_n(z) = \underbrace{\left( \prod_{j = n+1}^{\infty} (1+g_j(z)) - 1 \right)}_{(1)} \cdot \underbrace{\left( \prod_{j = 1}^{n} 1+g_j(z) \right)}_{(2)}. \]
For (1) apply \textbf{Claim 3} to conclude that
\[ \everymath{\displaystyle}
\arraycolsep=1.8pt\def\arraystretch{2.5}
\begin{array}{rcl}
    \left| \prod_{j = n+1}^{\infty} (1+g_j(z)) - 1 \right| & \leq & \exp \left( \sum_{k = n+1}^{\infty} |g_j(z)| \right) - 1.
\end{array} \]
Since $\sum_{k = n+1}^{\infty} |g_j(z)|$ converges uniformly to 0 (tail of a uniformly convergent sequence), for any $\varepsilon > 0$, there exists $N \in \N$ such that $\sum_{k = n+1}^{\infty} |g_j(z)| < \varepsilon$ for every $z \in X,\; n \geq N$, and thus,
\[ \exp \left( \sum_{k = n+1}^{\infty} |g_j(z)| \right) - 1 < \underbrace{e^{\varepsilon} - 1}_{\approx 0} \quad \forall z \in X,\; n \geq N. \]
so it follows that $\prod_{j = n+1}^{\infty} (1+g_j(z)) - 1$ converges uniformly to 0.

On the other hand, for (2), since $h := \sum_{j = 1}^{\infty} |g_j|$ is the uniform limit of continuous functions on a compact set, it follows that there exists $M > 0$ such that $h(z) < M$ for every $z \in X$. In fact, since $|g_j| \geq 0$, it follows that the sequence $h_n := \sum_{j = 1}^n|g_j|$ is increasing and $h_n \leq h < M$. Then, by \textbf{Claim 1}
\[ \everymath{\displaystyle}
\arraycolsep=1.8pt\def\arraystretch{2.5}
\begin{array}{rcl}
    \left|\prod_{j = 1}^{n} 1+g_j(z) \right| & = & \prod_{j = 1}^{n} |1+g_j(z)|\\
    & \leq & \prod_{j = 1}^{n} 1+|g_j(z)|\\
    & \leq & \exp\left( \sum_{j = 1}^n |g_j(z)| \right)\\
    & < & e^M\quad \forall z \in X.
\end{array}\]

Finally, (1) converges uniformly to 0 and (2) is uniformly bounded, so it follows that $|f_n(z) - f(z)|$ converges uniformly to 0.

\subsection*{Solution Item (b)}

Since $X$ is compact and $f$ is holomorphic, the number of zeros of $f$ is finite, otherwise, the set of zeros would have an accumulation point on $X$. If $\{z_1,\ldots, z_p\}$ is the set of zeros of $f$, the goal is to find $\forall j \leq p$, $N_j$ for which $g_{N_j}(z_j) = -1$ and then take $N = \max_j N_j$.

Let $z$ be one of those zeros and assume for the sake of contradiction that $g_j(z) \neq -1$ for every $j\in \N$. Then, we would obtain a contradiction with the fact that for a convergent product $\prod_{j= 1}^\infty a_j$ to be zero, one of the elements in the sequence is zero (otherwise, the product must diverge to 0), so let $a_j = (1+g_j(z))$ to obtain the contradiction. Therefore, for every $j$, there exists $N_j\in N$ for which $g_{N_j}(z_j) = -1$.

If $g_n(z) = -1$ for some $n \leq N$, then it's clear from the pointwise convergence that $z$ is a zero of $f$:
\[ f(z) = (1 - g_n(z)) \times \prod_{j\neq n} (1-g_j(z)). \]