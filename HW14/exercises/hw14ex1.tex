% chktex-file 3 chktex-file 9 chktex-file 15 chktex-file 16 chktex-file 17 chktex-file 18 chktex-file 36

\section{Exercise 1.}

Calculate
\[ \sum_{n = 1}^{\infty} n^{-4}. \]

\subsection*{Solution 1.}

\textbf{Note:} A time ago I found an interesting solution that only required to know that $\sum_{n = 1}^{\infty} n^{-2} = \frac{\pi^2}{6}$ and no calculus nor analysis. I won't prove some steps in this solution, the intended proof using complex analysis is at solution 2. If you want to skip it, just do it.

The first step is to define the following sequence

\[ a_{i,j} = \frac{2}{i^3 j} + \frac{1}{i^2 j^2} + \frac{2}{i j^3} .\]

Then, by expanding the sum $\sum_{i = 1}^{\infty} \sum_{j = 1}^{\infty} a_{i,j} - a_{i,i+j} - a_{i+j,j}$, there are multiple terms that are canceled:
\[ \begin{matrix}
    (\boldsymbol{a_{1,1}} \overset{(1)}{- a_{1,2} }\overset{(2)}{- a_{2,1} }) & + & (\overset{(1)}{a_{1,2}} \overset{(3)}{- a_{1,3} }\overset{(7)}{- a_{3,2} }) & + & (\overset{(3)}{a_{1,3} }\overset{(5)}{- a_{1,4}} - \cdots \hfill) & + & (\overset{(5)}{a_{1,4}}-\cdots\\
    (\overset{(2)}{a_{2,1} } \overset{(6)}{- a_{2,3} } \overset{(4)}{- a_{3,1} }) & + & (\boldsymbol{a_{2,2}} \overset{(9)}{- a_{2,4}} \overset{(10)}{-a_{4,2}}) & + & (\overset{(6)}{a_{2,3} }- \cdots \hfill ) & + & (\overset{(9)}{a_{2,4}}-\cdots\\
    (\overset{(4)}{a_{3,1} } \overset{(11)}{- a_{3,4}} \overset{(8)}{- a_{4,1}}) & + & (\overset{(7)}{a_{3,2} } - \cdots \hfill ) & + & (\boldsymbol{a_{3,3}} - \cdots \hfill ) & + & (\overset{(11)}{a_{3,4}}-\cdots\\
    (\overset{(8)}{a_{4,1}} - \cdots \hfill ) & + & (\overset{(10)}{a_{4,2}} - \cdots \hfill )
\end{matrix} \]
Eventually, the only terms that survive are in the diagonal $i = j$ (proof required), so 
\[ \sum_{i,j \in \N^+} a_{i,j} - a_{i,i+j} - a_{i+j,j} = \sum_{n = 1}^{\infty} a_{n,n} = \sum_{n = 1}^{\infty} \frac{5}{n^4}. \]
On the other hand, note that after simplifying (proof required), we obtain that
\[ a_{i,j} - a_{i,i+j} - a_{i+j,j} = \frac{2}{i^2 j^2}, \]
so by Cauchy summation,
\[ \sum_{i,j \in \N^+} a_{i,j} - a_{i,i+j} - a_{i+j,j} = \sum_{i,j \in \N^+} \frac{2}{i^2 j^2} = 2 \left( \sum_{i = 1}^{\infty} \frac{1}{i^2} \right)\cdot\left( \sum_{j = 1}^{\infty} \frac{1}{j^2} \right) = 2\left( \frac{\pi^2}{6} \right)^2. \]
Finally, after putting everything together, we obtain that
\[ \sum_{n = 1}^{\infty} \frac{5}{n^4} = \sum_{i,j \in \N^+} a_{i,j} - a_{i,i+j} - a_{i+j,j} = \frac{\pi^4}{18} \]
\[ \implies \sum_{n = 1}^{\infty} \frac{1}{n^4} =  \frac{\pi^4}{90}.\]

\subsection*{Solution 2.}

When $n\neq 0$, the $n$-th coefficient of the Fourier transform of $f(x) = x^2$ is
\[ \everymath{\displaystyle}
\arraycolsep=1.8pt\def\arraystretch{2.5}
\begin{array}{rcl}
    2\pi \cdot c_n & = &  \int_{-\pi}^{\pi} x^2 e^{-inx} dx\\
    (\mbox{\tiny $\textstyle uv-\int vdu$})& = & \underbrace{\left[ x^2 \cdot \frac{e^{-inx}}{-in} \right]_{-\pi}^{\pi}}_{A} + \frac{2}{in} \int_{-\pi}^{\pi} x e^{-inx} dx\\
    (\mbox{\tiny $\textstyle uv-\int vdu$})& = & A + \underbrace{\frac{2}{in}\left[ x \cdot \frac{e^{-inx}}{-in} \right]_{-\pi}^{\pi}}_{B} + \underbrace{\frac{2}{(in)^2} \int_{-\pi}^{\pi}  e^{-inx} dx}_{C}\\
\end{array} \]

Then, since $e^{\pm i n\pi} = (-1)^n$, it follows that
\[ A = \left( \pi^2 \cdot \frac{e^{-in\pi}}{-in} \right) - \left( (-\pi)^2 \cdot \frac{e^{in\pi}}{-in} \right) = \frac{\pi^2}{-in}((-1)^n - (-1)^n) = 0 \]
\[ B = \frac{2}{in} \left( \pi \cdot \frac{e^{-in\pi}}{-in} \right) - \frac{2}{in} \left( (-\pi) \cdot \frac{e^{in\pi}}{-in} \right) = \frac{2\pi}{n^2}((-1)^n + (-1)^n) = \frac{4\pi (-1)^n}{n^2} \]
\[ C = \frac{2}{(in)^2} \int_{-\pi}^{\pi}  e^{-inx} dx = \frac{2}{(in)^2} \cdot n \cdot \int_{\d B_{1}(0)} dz = 0. \]
\[ \implies c_n = \frac{A + B + C}{2\pi} = \frac{0+\frac{4\pi (-1)^n}{n^2} + 0}{2\pi} = \frac{2(-1)^n}{n^2} \]
\[ \implies |c_n|^2 = \frac{4}{n^4} \]
When $n = 0$,
\[ 2\pi \cdot c_0 = \int_{-\pi}^{\pi} x^2 dx = \frac{2\pi^3}{3} = 2\pi \cdot \frac{\pi^2}{3}. \]
\[ \implies c_0 = \frac{\pi^2}{3} \implies |c_0|^2 = \frac{\pi^4}{9}. \]

Then, using Parseval's Identity,
\[ \everymath{\displaystyle}
\arraycolsep=1.8pt\def\arraystretch{2.5}
\begin{array}{rcl}
    \frac{\pi^4}{9} + 8 \sum_{n = 1}^{\infty} \frac{1}{n^4} & = & |c_0|^2 + 2 \sum_{n = 1}^{\infty} |c_n|^2\\
    & = & \sum_{-\infty}^{\infty} |c_n|^2\\
    \mbox{\tiny (Parseval's Identity)}& = & \frac{1}{2\pi}\int_{-\pi}^{\pi} |x^2|^2 dx\\
    & = & \frac{1}{2\pi} \int_{-\pi}^{\pi} x^4 dx\\
    & = & \frac{1}{2\pi} \cdot \frac{2\pi^5}{5} = \frac{\pi^4}{5}
\end{array} \]
Finally, from this equality, it follows that
\[ \sum_{n = 1}^{\infty} \frac{1}{n^4} = \frac{1}{8} \left( \frac{\pi^4}{5} - \frac{\pi^4}{9} \right) = \frac{\pi^4}{8} \cdot \frac{4}{45} = \frac{\pi^4}{90}. \]