% chktex-file 3 chktex-file 9 chktex-file 17 chktex-file 18 chktex-file 36

\section{Introduction}

The goal of this project is to describe how the definition of holomorphic complex functions can be extended to linear operators in Banach spaces. For instance, if $P(z) = a_0 + a_1 z +\cdots + a_n z^n$ is a \textit{polynomial} defined for the complex numbers, then it's natural to define for a linear operator $T : X \to X$,
\[ P(T) = a_0 I + a_1 T + \cdots + a_n T^n, \]
\[ \underbrace{T^0 x := I x = x}_{\mbox{\tiny identity operator}},\quad \underbrace{T^k x = T(\cdots(T(x)))}_{\mbox{\tiny $k$ times composition}},\; \forall x \in X. \]

\begin{definition}\label{def:bounded_linear_operator}\label{def:operator_norm}
    A linear operator between two Banach spaces $T: X \to Y$ is called bounded if there exists $M > 0$ such that, for every $x \in X$,
    \[ \|Tx\|_Y \leq M \|x\|_X. \]
    The space of these functions is called $L(X,Y)$, and it's a Banach space if $Y$ is. Also, it can be shown that a linear operator is bounded if and only if is continuous. The operator's norm is defined as
    \[ \|T\| = \sup_{\|x\| = 1} \|Tx\|_Y. \]
\end{definition}

Now, for a holomorphic function $f: \C \to \C$ with a power series $f(z) = \sum_{n = 0}^{\infty} a_n z^n$ that absolutely converges for $|z| < R$ and a bounded linear operator $T$ such that $\|T\| < R$, the operator $f(T) := \sum_{n = 0}^{\infty} a_n T^n$ is well defined because
\[ \forall x \in X:\; \|f(T)(x)\| \leq \sum_{n = 0}^{\infty} |a_n| \|T^n x\| < \|x\| \sum_{n = 0}^{\infty} |a_n| R^n < \infty. \]
However, the tools provided by the power series are limited, and thus, we require another method that allow us to use similar versions of the theorems used in complex analysis
%  the clear limitation of this method is that we are restricted only to bounded operators. There are some notions of functional analysis that will be useful for extending this definition to a broader set of operators.

\begin{definition}\label{def:cauchy_domain}
    An open set $D\subset \C$ is said to be a \textit{Cauchy domain} if these two conditions are met:
    \begin{itemize}
        \item $D$ has a finite number of connected components.
        \item The boundary of $D$ is composed of a finite number of simple closed rectifiable curves that don't intersect with each other.
    \end{itemize}
\end{definition}
\begin{theorem}[Cauchy Integral Formula]\label{thm:cauchy_integral_formula_basic} 
    Let $U\subset \C$ be an open set and $f:U\to \C$ a holomorphic function. Let $D$ be a Cauchy domain such that $\ol{D} \subseteq U$. Then,
    \[ \int_{C} f(\lambda) d\lambda = 0, \]
    where $C$ denotes any positively oriented curve that encloses the boundary of $D$. Also, the $n$-th derivative of $f$, $f^{(n)}$ exists on $D$, and for $\lambda_0 \in D$,
    \[ f^{(n)}(\lambda_0) = \frac{n!}{2\pi i} \int_{C} \frac{f(\lambda)}{(\lambda - \lambda_0)^{n+1}} \, d\lambda. \] 
    \hfill $\square$
\end{theorem}

Therefore, we find a more general way to define $f(T)$ using a version of the Cauchy Integral formula,
\[ f(T) = \frac{1}{2\pi i} \int_{C} \frac{f(\lambda)}{\lambda - T} \; d \lambda, \]
where $1/(\lambda - T) := (\lambda I - T)^{-1}$ is called the \textit{resolvent} of $T$ at $\lambda$ and it's defined only if the operator $(\lambda I - T)$ is \textit{invertible}. The scope of this project is to formalize this particular definition of the Cauchy Formula, which requires the introduction of a more general notion of complex differentiation and integration for functions with values on Banach spaces.

\begin{definition}\label{def:resolvent}\label{def:spectrum}
    The \textit{resolvent set} of a linear operator $T: X \to X$ in a Banach space $X$ is defined as
    \[ \rho(T) := \{\lambda \in \C \;:\; (\lambda I - T) \mbox{ is bijective}\}. \]
    If $\lambda \in \rho(T)$, then the \textit{resolvent of $T$ at $\lambda$} is $R(\lambda , T) := (\lambda I - T)^{-1}$. Also, the complement of the resolvent set, $\sigma(T) := \C \backslash \rho(T)$ is called the \textit{spectrum of $T$}.
\end{definition}

% The following theorem is used some times in the following propositions, but the proof might be out of the scope of this document, so we're just going to state it.

% \begin{theorem}\label{thm:resolvent_open_for_bounded}
%     For a Banach space $X$ and a bounded linear operator $T: X \to X$, the \textit{resolvent set} $\rho(T) \subset \C$ is an open set, and thus, the spectrum $\sigma(T)$ is closed.
%     $ $\hfill $\square$
% \end{theorem}


