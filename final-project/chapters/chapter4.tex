% chktex-file 3 chktex-file 9 chktex-file 12 chktex-file 17 chktex-file 18 chktex-file 36
\section{The Resolvent}

To conclude the discussion we are going to describe some properties of the resolvent operator. Similar to the discussion in~\cite{taylor1986introduction}, for this section, we use the notation $T_\lambda = (\lambda I - T)$ and $R_\lambda = R(\lambda; T)$. Using Lemma 3.33 from~\cite{winklmeier2013functional}, we can assert that if $T:X\to X$ is a closed operator and $\lambda \in \rho(T)$, then $R_\lambda$ is bounded. So from this point forward, assume that $T: X \to X$ is a closed linear operator. 

In the first place, for $S \in L(X)$ note that if $\|S\| < 1$, the Neumann series gives us
\[ (I-S)^{-1} = \sum_{n = 0}^{\infty} S^n. \]
Now, for $\lambda_0 \in \rho(T)$ and $\lambda \in \C$, we have that since 
\[ T_{\lambda} = T_{\lambda_0} - (\lambda_0 - \lambda) I = (I - (\lambda_0-\lambda)R_{\lambda_0})T_{\lambda_0}. \]
So if we choose $\lambda$ such that $\|(\lambda_0-\lambda)R_{\lambda_0}\| < 1$, that is $|\lambda_0-\lambda|< \|R_{\lambda_0}\|^{-1}$, then we can invert the previous expression using the Neumann series to obtain
\[ \everymath{\displaystyle}
\arraycolsep=1.8pt\def\arraystretch{2.5}
\begin{array}{rcl}
    R_{\lambda} & = & R_{\lambda_0} (I - (\lambda_0-\lambda)R_{\lambda_0})^{-1}\\
    & = & R_{\lambda_0} \sum_{n = 0}^{\infty} (\lambda_0-\lambda)^n R_{\lambda_0}^n\\
    & = & \sum_{n = 0}^{\infty} (\lambda_0-\lambda)^n R_{\lambda_0}^{n+1}.
\end{array} \]

This proves that for $\lambda_0 \in \rho(T)$, the ball of radius $\|R_{\lambda_0}\|^{-1}$ is contained at $\rho(T)$, and thus, the resolvent set is open. On the other hand, note that if $T$ is bounded and $|\lambda| > \|T\|$, then $\|\lambda^{-1} T\| < 1$, so it follows that
\[ T_{\lambda} = \lambda(I - \lambda^{-1} T) \mbox{ is invertible.} \]
Therefore, $\sigma(T) \subseteq \ol{B_{\|T\|}(0)}$ implying that $\sigma(T)$ is compact when $T \in L(X)$. 

\begin{theorem}\label{thm:resolvent_is_open}
    The resolvent of a closed linear operator is open. Also, for $\lambda_0 \in \rho(T)$, if $|\lambda_0 - \lambda| < \|R_{\lambda_0}\|^{-1}$, then $\lambda \in \rho(T)$.\hfill $\square$
\end{theorem}


In the next theorem, we are going to prove that the mapping $\lambda \mapsto R_\lambda$ is analytic in its domain.

\begin{theorem}
    For $y\in X$ and $x' \in X'$, the mapping $\lambda \mapsto x'R_{\lambda}y$ is holomorphic.
\end{theorem}

\begin{proof}
    Let $\lambda_0$. Then, note that the Neumann series provided by \hyperref[thm:resolvent_is_open]{Theorem~\ref*{thm:resolvent_is_open}} is absolutely convergent for $\lambda \in B_{\|R_{\lambda_0}\|^{-1}}(\lambda_0)$. Thus, by continuity of $x'$,
    \[ \everymath{\displaystyle}
    \arraycolsep=1.8pt\def\arraystretch{2.5}
    \begin{array}{rcl}
        x'R_{\lambda}y & = & x'\left( \sum_{n = 0}^{\infty} (\lambda_0-\lambda)^n R_{\lambda_0}^{n+1} y \right)\\
        & = & \sum_{n = 0}^{\infty} (\lambda_0-\lambda)^{n} \underbrace{x' R_{\lambda_0}^{n+1}y}_{a_n}\\
        & = & \sum_{n = 0}^{\infty} a_n (\lambda_0-\lambda)^{n}.
    \end{array} \]
\end{proof}

\begin{corollary}\label{cor:resolvent_analytic} The following consequences follow from the previous theorem. For $\lambda \in \rho(T)$,
    \begin{itemize}
        \item  The mapping $\lambda \mapsto R_{\lambda}$ is analytic (as an operator in $L(X)$).
        \item  For a holomorphic function $f$, the mapping $\lambda \mapsto f(\lambda)R_{\lambda}$ is analytic in any open set at the complement of $\sigma^2$ (as an operator in $L(X)$).
        % \item The $n$-th derivative of $R_{\lambda}$ exists for $n = 1,2,\ldots$,
        % \[ \frac{d^n}{d \lambda^n} R_\lambda = (-1)^n n! R_{\lambda}^{n+1}. \]
    \end{itemize}
    \hfill $\square$
\end{corollary}

Therefore, the following theorem follows from the fact that $\sigma(T)$ is compact for a bounded operator $T$.

\begin{theorem}\label{thm:cauchy_integral_well_defined}
    Let $f: U \to \C$ be a complex function that is holomorphic at a bounded Cauchy domain $D \supseteq \sigma(T)$ and let $C_1,C_2$ be two rectifiable simple curves that form the boundary of two open sets $U_1, U_2 \subseteq D$ such that $U_1, U_2 \supseteq \sigma(T)$. Then,
    \[ \int_{C_1} f(\lambda) R_\lambda\, d\lambda = \int_{C_2} f(\lambda) R_\lambda\, d\lambda. \]
\end{theorem}

\begin{proof}
    $C_1,C_2$ are compact. Let $-C_2$ be the negatively oriented version of $C_2$. If we join $C_1$ and $-C_2$, then, we obtain another compact curve $C = C_1-C_2$ that can be contained at an open set $U$ that is at the complement of $\sigma(T)$. Therefore, by \hyperref[cor:resolvent_analytic]{Corollary~\ref*{cor:resolvent_analytic}} since the function $\lambda \mapsto \lambda f(\lambda)R_\lambda$ is analytic at $U$, it follows by the Cauchy Integral Formula that
    \[ 0 = \int_{C} f(\lambda)R_\lambda\, d\lambda = \int_{C_1} f(\lambda)R_\lambda\, d\lambda - \int_{C_2} f(\lambda)R_\lambda\, d\lambda \]
\end{proof}

\subsection*{Closing statements}

From this discussion we justified the methods to calculate the Cauchy Integral for bounded operators. When $\sigma(T)$ is contained at a bounded Cauchy domain $D$ where $f: U \to \C$ is holomorphic,
\[ f(T) = \int_{\d D} f(\lambda) R_\lambda\, d\lambda, \]
is well defined and independent of the choice of $D$. Also since we proved that, for a bounded operator $T$, the spectrum is bounded, we can find $R > 0$ such that $R(\lambda; T)$ is defined,
\[ R(\lambda; T) = \sum_{n = 1}^{\infty} \lambda^{-n} T^{n-1},\quad |\lambda| > R. \]
There are many important repercussions that require further investigation. For example, the next consequence of this integral formula would be the following theorems

\begin{theorem}
    \[ T^p := \int_{\d D} \lambda^{p}R_\lambda d\lambda,\quad p = 0,1,\ldots. \]
    \hfill $\square$
\end{theorem}

\begin{theorem}
    Let $f_1,f_2$ be holomorphic functions that satisfy the conditions stated in \hyperref[thm:cauchy_integral_well_defined]{Theorem~\ref*{thm:cauchy_integral_well_defined}}. Then, $(\alpha f_1 + f_2)(T) = \alpha f_1(T) + f_2(T)$ and $(f_1f_2)(T) = f_1(T)f_2(T)$.\hfill $\square$
\end{theorem}

\begin{theorem}
    Let $T \in L(X)$. If $f,g$ are two holomorphic function defined on an open set that contains $\sigma(T)$ and $f(\lambda) = g(\lambda)$ for every $\lambda \in U$ for an open set $U$ contained in their domains, then $f(T) = g(T)$.\hfill $\square$
\end{theorem}

For unbounded closed operators, since $\sigma(T)$ might not be bounded, more details have to be taken into consideration. A plausible alternative is to only consider a set of functions $\F(f)$ that satisfies the following conditions:
\begin{itemize}
    \item For $f \in \F(T)$, $f$ is holomorphic on its domain $U\subset \C$ which is an open set that contains $\sigma(T)$.
    \item The complement of $U$ is compact.
    \item $f(\lambda)$ is bounded and the limit $f(\infty) = \lim_{|\lambda|\to\infty}f(\lambda)$ exists.
\end{itemize}

In fact, for $f\in \F(T)$ with $f: U \to \C$, if $D$ is an unbounded Cauchy domain such that $\ol{D} \subset U$, then $\d B$ is compact and the Cauchy Integral Formula would hold as follows
\[ f(\lambda_0) = f(\infty) + \int_{\d D}  \frac{f(\lambda)}{\lambda_0-\lambda}\, d\lambda. \]
Therefore, according to the previous discussion, the appropriate definition of $f(T)$ would be the following
\[ f(T) :=  f(\infty) I + \int_{\d D} f(\lambda) R_\lambda\, d\lambda.\]
From this point forward, many of the results for bounded operators, follow similarly with closed operators.
% \[ T^p = \frac{1}{2\pi i} \int_C \lambda^p R_\lambda d\lambda,\quad  p = 0,1,2,\ldots\] 
% Therefore, 


% \begin{proof}
%     The first 2 items follow directly from the previous theorem. For the last item we require an identity
%     \begin{lemma}[Resolvent Infodentity]\label{lem:resolvent_identity}
%         \[ R_{\lambda} - R_{\mu} = (\mu-\lambda)R_{\lambda}R_{\mu},\quad \lambda,\mu \in \rho(T). \]
%     \end{lemma}
%     \begin{proof}
%         \[ \everymath{\displaystyle}
%         \arraycolsep=1.8pt\def\arraystretch{1.5}
%         \begin{array}{rcl}
%             R_{\lambda} - R_{\mu} & = & R_{\lambda}(I- T_\lambda R_\mu)\\
%             & = & R_{\lambda}(T_\mu - T_\lambda)R_\mu\\
%             & = & R_{\lambda}(\mu - T - \lambda + T)R_\mu\\
%             & = & (\mu - \lambda)R_{\lambda}R_\mu
%         \end{array} \]
%     \end{proof}
% \end{proof}
