% chktex-file 3 chktex-file 9 chktex-file 17 chktex-file 18 chktex-file 36

\section{Complex Derivatives for Vector Valued Functions}

The goal of this section is to extend the notion of analyticity to a broader set of functions. We're basing this discussion mainly on Chapter V of~\cite{taylor1986introduction}, but some definitions and proofs were adapted from Chapter 5 of~\cite{winklmeier2013functional}.

Let $X$ be a complex Banach space, and let $f: U\subseteq \C \to X$ be a function that is defined on an open set $U$ with values in $X$.

\begin{definition}\label{def:analytic}
    We say that $f: U \to X$ is \textit{analytic} on $U$ if for every $\lambda_0 \in U$, there exists an element of $X$ called $f'(\lambda_0)$ such that
    \[ \lim_{\lambda \to \lambda_0} \left\| \frac{f(\lambda_0)-f(\lambda)}{\lambda_0 - \lambda} - f'(\lambda_0) \right\| = 0. \]
\end{definition}

In order for the previous definition to be relevant with our problem, there must exist a bridge between functions from $\C \to X$ and holomorphic functions $\C \to \C$. The bridge in mind is the dual space of $X$, which provides an interesting way to translate analyticity in this context to complex differentiability.

\begin{definition}\label{def:dual_space}\label{def:functional}
    Let $X$ be a Banach space. We define $X' = L(X,\C)$ as the dual space of $X$, which consists of all bounded operators from $X$ to $\C$. The elements of the dual space are called \textit{functionals}. Similarly to how we treat operators in general, for $x'\in X'$ and $y \in X$, we use the notation
    \[ x'y = x'(y) .\]
    For $x' \in X'$, the norm $\|x'\|$ is defined as the operator's norm of $x'$.
\end{definition}

\begin{remark} Since functionals are continuous linear functions, it follows that for an analytic function $f$ and $x' \in X$,
    \[ \lim_{\lambda \to \lambda_0} \frac{x'f(\lambda_0)-x'f(\lambda)}{\lambda_0 - \lambda} = x'\left( \lim_{\lambda \to \lambda_0} \frac{f(\lambda_0)-f(\lambda)}{\lambda_0-\lambda}  \right) = x'f'(\lambda_0). \]
Therefore, if $f$ is analytic, then $x'f := x'\circ f: U \to \C$ is an holomorphic function. 
\end{remark}

In function analysis, we often refer to a \textit{weak} version of a property when the property holds when evaluated using functionals in the dual space. For example, a sequence $\{x_n\}_{n\in \N} \subset X$ is said to \textit{weakly converge} to $x\in X$ if, for every $x'\in X$, the sequence $x'x_n$ converges to $x'x$. With the previous remark, we could introduce a weak version of analyticity when $x'f'(\lambda_0)$ exists for every $\lambda_0 \in U$ and $x' \in X'$. However, part of what makes everything work here is that this weak notion is equivalent to analyticity.

\begin{theorem}\label{thm:weakly_analytic_implies_analytic}
    If $x'f: U\to \C$ is holomorphic for every $x'\in X$, then $f: U \to X$ is analytic.
\end{theorem}

\begin{proof}
    Since $X$ is a complete space, it suffices to prove that for every $\lambda_0 \in U$,
    \[ \frac{f(\lambda_0)-f(\lambda)}{\lambda_0 - \lambda} - \frac{f(\lambda_0)-f(\mu)}{\lambda_0 - \mu} \to 0,\quad \mbox{when $\lambda,\mu \to \lambda_0$ on any direction.} \]
    Before proceeding, we must state a theorem that is important for this proof.
    \begin{lemma}[Uniform Boundness Principle]\label{lem:uniform_boundness_principle}
        Let $A$ be a subset of a linear space $X$ with a norm $\|\cdot\|$. If, for every $x'\in X'$, $x'A = \{x'a \;:\; a \in A\}$ is bounded, then there exists $M >0$ such that $\|a\| \leq M$ for every $a \in A$. In other words, if $x'A$ is bounded for every $x'\in X$, then $A$ is bounded too.\hfill $\square$
    \end{lemma}

    Let $r > 0$ and let $C := \ol{B_{r}(\lambda_0)}$ be a positively oriented circe centered at $\lambda_0$ with radius $r$. Since $x'f := x'\circ f$ is a holomorphic function (and thus continuous), it follows that $x'f(C)$ is a bounded set for every $x'\in X'$. Therefore, by the \hyperref[lem:uniform_boundness_principle]{Uniform Boundness Principle}, there exists $M > 0$ such that
    \[ \|f(\lambda)\|_X \leq M,\quad \lambda \in C. \]
    Now, for $\lambda,\mu \in B_{r/2}(\lambda_0)$, \hyperref[thm:cauchy_integral_formula_basic]{Cauchy Integral Formula} give us
    \[ x'f(\lambda) = \frac{1}{2\pi i}\int_{C} \frac{x' f(\zeta)}{\zeta - \lambda} d\zeta, \]
    so we have that,
    \[ \everymath{\displaystyle}
    \arraycolsep=1.8pt\def\arraystretch{2.5}
    \begin{array}{rcll}
        x'\left[ \frac{f(\lambda_0)-f(\lambda)}{\lambda_0 - \lambda} - \frac{f(\lambda_0)-f(\mu)}{\lambda_0 - \mu} \right] & = & & \frac{1}{2\pi i} \int_C \frac{x'f(\zeta)}{(\lambda_0 - \lambda)(\zeta - \lambda_0)} - \frac{x'f(\zeta)}{(\lambda_0 - \lambda)(\zeta - \lambda)} d\zeta\\
        & & + & \frac{1}{2\pi i} \int_C \frac{x'f(\zeta)}{(\lambda_0 - \mu)(\zeta - \lambda_0)} - \frac{x'f(\zeta)}{(\lambda_0 - \mu)(\zeta - \mu)} d\zeta\\
        & = & & \frac{1}{2\pi i} \int_C \frac{(\lambda-\mu) x'f(\zeta) }{(\zeta - \lambda_0)(\zeta - \lambda)(\zeta - \mu)} d\zeta.
    \end{array} \]
    Now note that since $|\zeta - \lambda| \geq  r/2$, $|\zeta - \mu| \geq r/2$ and $|\zeta - \lambda_0| = r$, it follows that
    \[ \everymath{\displaystyle}
    \arraycolsep=1.8pt\def\arraystretch{2.5}
    \begin{array}{rcl}
        \left| \frac{1}{2\pi i} \int_C \frac{(\lambda-\mu) x'f(\zeta)}{(\zeta - \lambda_0)(\zeta - \lambda)(\zeta - \mu)} d\zeta \right| & \leq & \frac{4|\lambda-\mu|}{2\pi r^3} \sup_{\lambda \in C} |x' f(\lambda)| \cdot \underbrace{\mbox{len}(C)}_{2\pi r} \\
        & \leq & \frac{4 |\lambda-\mu|}{r^2} M \|x'\|.    \end{array}\]
    The last inequality follows from the fact that $\|x'y\| \leq \|x'\| \|y\|$. Again, before concluding this proof, we require another theorem from functional analysis.
    \begin{lemma}[Hahn-Banach Theorem]\label{lem:hahn_banach_thm} If $y \neq 0 \in X$, then there exists $x' \in X'$ such that $\|x'\| = 1$ and $x'y = \|x\|$. In particular, if for every $x'\in X'$, $x'y = 0$, then $y = 0$.\hfill $\square$
    \end{lemma}
    Finally, we conclude that there exists some $x' \in X'$ with $\|x'\| = 1$ and
    \[ \everymath{\displaystyle}
    \arraycolsep=1.8pt\def\arraystretch{2.5}
    \begin{array}{rcl}
        \left\|  \frac{f(\lambda_0)-f(\lambda)}{\lambda_0 - \lambda} - \frac{f(\lambda_0)-f(\mu)}{\lambda_0 - \mu} \right\| & = & \left| x'\left(  \frac{f(\lambda_0)-f(\lambda)}{\lambda_0 - \lambda} - \frac{f(\lambda_0)-f(\mu)}{\lambda_0 - \mu} \right) \right| \\
        & \leq &  \frac{4M}{r^2} \cdot |\lambda-\mu| \to 0,\quad \lambda,\mu \to \lambda_0.
    \end{array}\]
    so we conclude that $f$ is analytic.
\end{proof}

From the previous discussion it is relatively easy to see that all notions of analyticity for maps $U\subseteq \C \to L(X,Y)$ are equivalent.

\begin{corollary}\label{cor:analytic_equivalences}
    Let $X, Y$ be Banach spaces and let $U \subseteq \C$ be an open set. For every $\lambda \in U$, let $A_\lambda \in L(X,Y)$. The following statements are equivalent:
    \begin{enumerate}[label=(\alph*)]
        \item For $x\in X,\; y'\in Y'$, the mapping $\lambda \mapsto y'(A_\lambda x)$ is analytic (\textit{weak analyticity}).
        \item For $x\in X'$, the mapping $\lambda \mapsto A_\lambda x$ is analytic (\textit{strong analyticity}).
        \item The mapping $\lambda \mapsto A_\lambda$ is analytic (\textit{analytic in the operator norm}).
    \end{enumerate}
\end{corollary}

\begin{proof}
    (a) $\iff$ (b) follows from \hyperref[thm:weakly_analytic_implies_analytic]{Theorem~\ref*{thm:weakly_analytic_implies_analytic}}.~(c) $\implies$ (b) follows from the definition:
    \[ \left\| \frac{A_{\lambda_0}x - A_\lambda x}{\lambda_0 - \lambda} \right\| \leq \left\| \frac{A_{\lambda_0} - A_\lambda}{\lambda_0 - \lambda} \right\| \|x\| \to 0,\quad \lambda\to \lambda_0,\;\forall x \in X. \]
    Finally, for (a) $\implies$ (c), suppose that $\lambda \mapsto y' A_\lambda x$ is analytic (and thus continuous) for every $x \in X$. Then, since $C$ is a compact set, by the \hyperref[lem:uniform_boundness_principle]{Uniform Boundness Principle} there exists $M_x > 0$ such that 
    \[ \sup_{\lambda \in C} |A_\lambda x| \leq M_x. \]
    We need the following theorem to proceed,
    \begin{lemma}[Banach-Steinhaus theorem]\label{lem:banach_steinhaus}
        Let $X$ be a Banach space and $Y$ a normed space. Let $F \subset L(X,Y)$ be a family of operators such that
        \[  \forall x \in X,\; \exists M_x > 0\quad \forall f\in F\; \|f(x)\| \leq M_x. \]
        Then, there exists $M > 0$ such that
        \[ \|f\| < M\quad \forall f \in F \]
        \hfill $\square$
    \end{lemma}
    Using this lemma, we obtain $M > 0$ such that
    \[ \sup_{\lambda \in C} |A_\lambda x| \leq M \|x\|,\quad \forall x\in X. \]
    Finally, similar to the proof of the previous theorem,
    \[ \left\| \frac{A_{\lambda_0}x - A_{\lambda} x}{\lambda_0 - \lambda} \right\| \leq \frac{4 |\lambda-\mu|}{r^2} \sup_{\lambda \in C} |A_\lambda x| . \]
    Thus,
    \[ \left\| \frac{A_{\lambda_0} - A_{\lambda}}{\lambda_0 - \lambda} \right\| \leq  \frac{4 |\lambda-\mu|}{r^2} M \to 0\quad \lambda,\mu \to \lambda_0.\]
\end{proof}

