% chktex-file 18 chktex-file 36
\section*{Exercise 1.}

Find all the points $z\in\C$ where the following functions are differentiable and find the largest open set $U$ where they are holomorphic.
\begin{enumerate}[label=(\alph*)]
    \item $f(z) = \ol{z}$
    \item $ f(x+iy) = x^2 + y^2 + i(x^2 - y^2)$
\end{enumerate}

\subsection*{Solution Part (a)}

\[ f(x+iy) = x-iy = u(x,y) + i v(x,y) \]
Then,
\[ \frac{\d u}{\d x}(x,y) = 1,\hspace{2em} \frac{\d v}{\d x}(x,y) = 0, \]
\[ \frac{\d u}{\d y}(x,y) = 0,\hspace{2em} \frac{\d v}{\d y}(x,y) = -1, \]
All the partial derivatives exists and are continuous on any $(x,y)\in\R^2$, and thus, the function is differentiable. However, the Cauchy-Riemann equations are a requirement for $f$ to be complex-differentiable. Therefore, since $\d u/\d x \neq \d v/\d y$ on all points, the largest open set where it's holomorphic is $U = \emptyset$.

\subsection*{Solution Part (b)}

In this case,
\[ u(x,y) = x^2 + y^2,\hspace{1em} v(x,y) = x^2-y^2, \]
and the respective partial derivatives are
\[ \frac{\d u}{\d x} = 2x, \hspace{1em} \frac{\d v}{\d x} = 2x \]
\[ \frac{\d u}{\d y} = 2y, \hspace{1em} \frac{\d v}{\d y} = -2y. \]
For differentiability in $\R^2$, the argument is again that the partial derivatives exist and are continuous. For complex-differentiability, the function is holomorphic only when $2x = -2y$. Thus, the largest open set is again $U = \emptyset$.