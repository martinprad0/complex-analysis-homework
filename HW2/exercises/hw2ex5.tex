% chktex-file 18 chktex-file 36
\section*{Exercise 5.}

A subset $S\subset \N$ is in \textit{arithmetic progression} if there exists $a,d\in\N$ such that
\[ S = \{a+nd \;:\; n\in\N_0\}. \]
The number $d$ is called the difference of the progression. Prove that $\N$ cannot be partitioned in a finite number greater than 1 of arithmetic progressions with different differences.

\textbf{Solution:}

Let $A_i = \{a_i + n d_i\}$ with $d_i \neq d_j$ if $i\neq j$. Assume that there exists a finite partition of $\N^+$ $\{A_1,\ldots, A_N\}$. W.L.O.G. assume that $d_1 < d_2 < \cdots < d_N$. Then,
\[ \sum_{n = 1}^{\infty} z^n = \sum_{n = 1}^{\infty}z^{a_1+n d_1} + \cdots + \sum_{n = 0}^{\infty}z^{a_N+n d_N}. \]
This power series converges when $|z| < 1$:
\[ \sum_{n = 1}^{\infty} z^n = \frac{z}{1-z} = \sum_{i = 1}^N \frac{z^{a_i}}{1-z^{d_i}} \]

Note that $\frac{1}{1-z}$ has a pole of multiplicity 1 at $z = 1$. However, the right-hand side of the equation has a pole of multiplicity $d_N > 1$ at $z = 1$ and this would give us a contradiction.