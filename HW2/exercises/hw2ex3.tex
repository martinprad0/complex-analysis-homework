% chktex-file 3 chktex-file 18 chktex-file 36 chktex-file 40
\section*{Exercise 3.}

\begin{enumerate}[label=(\alph*)]
    \item $\exp(z+w) = \exp(z) \exp(w)$.
    \item $\exp(z) \neq 0$ for all $z\in\C$.
    \item $|\exp(z)| = 1$ if and only if $z\in i\R$.
    \item $\cos^2(z)+\sin^2(z) = 1$ for all $z \in \C$.
    \item $\cos(z+ 2\pi) = \cos(z)$ and $\sin(z+2\pi) = \sin(z)$ for all $z \in \C$.
    \item $\cos(z) = 0$ or $\sin(z) = 0$ $\implies$ $z\in\R$.
    \item For every $x\in\R$, $\lim_{t\to\pm\infty}|\cos(x+it)| = \infty $ and $\lim_{t\to\pm\infty}|\sin(x+it)| = \infty $. The limit is uniform in $x$.
\end{enumerate}

\subsection*{Solution Part (a)}
The Cauchy product of 2 series implies that
\[ \everymath{\displaystyle}
\arraycolsep=1.8pt\def\arraystretch{2.5}
\begin{array}{rcl}
    \exp (z) \exp (w) & = & \left( \sum_{n = 0}^{\infty} \frac{z^n}{n!} \right) \left( \sum_{m = 0}^{\infty} \frac{w^m}{m!} \right)\\
    & = & \sum_{n = 0}^{\infty} \sum_{k = 0}^n \frac{z^k w^{n-k}}{k!(n-k)!}\\
    & = & \sum_{n = 0}^{\infty} \sum_{k = 0}^n \frac{z^k w^{n-k}}{k!(n-k)!} \cdot \frac{n!}{n!}\\
    & = & \sum_{n = 0}^{\infty} \frac{1}{n!} \sum_{k = 0}^n \binom{n}{k} z^k w^{n-k}\\
    & = & \sum_{n = 0}^{\infty} \frac{(z+w)^n}{n!}\\
    & = & \exp(z+w)
\end{array} \]

\subsection*{Solution Part (b)}

For every complex number $z$, there exists an additive inverse $(-z)$ such that
\[ z + (-z) = 0. \]
Thus, if it was the case that there exists $z\in \C$ such that $e^z = 0$, then, using part (a),
\[ 1 = e^0 = e^{z + (-z)} = e^{z} e^{-z} = 0, \]
and this would lead to a contradiction.

\subsection*{Solution Part (c)}

$\boldsymbol{\impliedby:}$ We are going to prove Euler's formula for the power series definition. Let $z = iy,\; y\in \R$,
\[ \everymath{\displaystyle}
\arraycolsep=1.8pt\def\arraystretch{2.5}
\begin{array}{rcl}
    \exp(iy) & = & \frac{(iy)^0}{0!} + \frac{(iy)^1}{1!} + \frac{(iy)^2}{2!} + \frac{(iy)^3}{3!} + \cdots\\
    & = & \left( \frac{(iy)^0}{0!} + \frac{(iy)^2}{2!} + \cdots \right) + i\left( \frac{i^0 y^1}{1!} + \frac{i^2 y^3}{3!} + \cdots \right)\\
    & = & \sum_{n = 0}^{\infty} i^{2n}\frac{y^{2n}}{(2n)!} + i \sum_{n = 0}^{\infty} i^{2n}\frac{y^{2n+1}}{(2n+1)!}\\
    & = & \sum_{n = 0}^{\infty} (-1)^{n}\frac{y^{2n}}{(2n)!} + i \sum_{n = 0}^{\infty} (-1)^{n}\frac{y^{2n+1}}{(2n+1)!}\\
    & = & \cos(y) + i \sin(y).
\end{array} \]
Note that from this and part (a), it follows that for $x,y\in\R$,
\[ e^{x+iy} = e^x (\cos(y) + i \sin(y)) \]
Therefore, we have $\cos^2(y) + \sin^2(y) = 1$ for $y\in \R$, and thus,
\[ |\exp(iy)| = \sqrt{\cos^2(y) + \sin^2(y)} = 1 \]

$\boldsymbol{\implies:}$ Let $z = x+iy,\; x,y\in\R$ such that $|\exp(z)| = 1$. Then, using part (a), $|\exp(z)| = |\exp(x)| |\exp(iy)|$. 
Then, using the previous implication, we know that $|\exp(iy)| = 1$. 
Therefore, $|\exp(z)| = |\exp(x)| = \exp(x) = 1$, but for real numbers, the only solution for $\exp(x) = 1$ is $x  = 0$.

\subsection*{Solution Part (d)}
According to Ahlfors' book, the definition of the cosine and sine functions are:
\[ \cos(z) = \frac{e^{iz}+e^{-iz}}{2},\hspace{1em}\sin(z) = \frac{e^{iz}-e^{-iz}}{2i}.  \]
Thus,
\[ \everymath{\displaystyle}
\arraycolsep=1.8pt\def\arraystretch{2.5}
\begin{array}{rcl}
    \cos^2(z) + \sin^2(z) & = & \left( \frac{e^{iz}+e^{-iz}}{2} \right)^2 + \left( \frac{e^{iz}-e^{-iz}}{2i}  \right)^2 \\
    & = & \frac{1}{4} e^{2iz} + \frac{e^{iz-iz}}{2} + \frac{1}{4} e^{-2iz} - \frac{1}{4} e^{2iz} + \frac{e^{iz-iz}}{2} - \frac{1}{4} e^{-2iz}\\
    & = & e^{iz-iz} = 1.
\end{array} \]

\subsection*{Solution Part (e)}
Using Euler's formula, we know that $e^{-2\pi i} = e^{2\pi i} = \cos(2\pi) + i\sin(2\pi) = 1$. Now, with the same identity and part (a),
\[ \everymath{\displaystyle}
\arraycolsep=1.8pt\def\arraystretch{2.5}
\begin{array}{rcl}
    \cos(z+2\pi) & = & \frac{e^{iz+i2\pi}+e^{-iz - i2\pi}}{2}\\
    & = & \frac{e^{iz}e^{2\pi i}+e^{-iz }e^{-2\pi i}}{2}\\
    & = & \frac{e^{iz}+e^{-iz}}{2} = \cos(z),
\end{array} \]
\[ \everymath{\displaystyle}
\arraycolsep=1.8pt\def\arraystretch{2.5}
\begin{array}{rcl}
    \sin(z+2\pi) & = & \frac{e^{iz+i2\pi}-e^{-iz - i2\pi}}{2i}\\
    & = & \frac{e^{iz}e^{2\pi i}-e^{-iz }e^{-2\pi i}}{2i}\\
    & = & \frac{e^{iz}-e^{-iz}}{2i} = \sin(z).
\end{array} \]

\subsection*{Solution Part (f)}
Let $z = x+iy$. Either $\cos(z) = 0$ or $\sin(z) = 0$ (or both). So let's start with the case $\cos(z) = 0$. Using general Euler's formula,
\[ e^{iz} = \cos(z) + i \sin (z) = i\sin(z) =  \frac{e^{iz}-e^{-iz}}{2} \]
Remember that for $x\in\R,\; |e^{ix}| = 1$, and $|e^x| = e^x$. Now,
\[ \everymath{\displaystyle}
\arraycolsep=1.8pt\def\arraystretch{1.5}
\begin{array}{rrcl}
    \implies & e^{-iz} & = & -e^{iz}\\
    \implies & e^{y-ix} & = & -e^{-y+ix}\\
    \implies & |e^{-ix}||e^{y}| & = & |-1||e^{ix}||e^{-y}|\\
    \implies & e^y & = & e^{-y}\\
    \implies & e^{2y} & = & 1\\
    \implies & y & = & \im(z) = 0.
\end{array}\]
For the case when $\sin(z) = 0$, the argument is a similar one:
\[ e^{iz} = \cos(z) + i \sin (z) = \cos(z) =  \frac{e^{iz}+e^{-iz}}{2} \]
\[ \everymath{\displaystyle}
\arraycolsep=1.8pt\def\arraystretch{1.5}
\begin{array}{rrcl}
    \implies & e^{-iz} & = & e^{iz}\\
    \implies & e^{y-ix} & = & e^{-y+ix}\\
    \implies & |e^{-ix}||e^{y}| & = & |e^{ix}||e^{-y}|\\
    \implies & e^y & = & e^{-y}\\
    \implies & e^{2y} & = & 1\\
    \implies & y & = & \im(z) = 0.
\end{array}\]

\subsection*{Solution Part (g)}
\[ \everymath{\displaystyle}
\arraycolsep=1.8pt\def\arraystretch{2.5}
\begin{array}{rcl}
    |\cos(x+it)| & = & \left| \frac{e^{ix-t}+e^{-ix+t}}{2} \right|\\
    & = & \left| \frac{e^{ix}e^{-t}+e^{-ix+t}e^{t}}{2} \right|\\
    \mbox{($\triangle$-ineq)} & \geq & \frac{1}{2} \left| |e^{ix}||e^{-t}| - |e^{-ix}||e^{t}| \right|\\
    & = & \frac{1}{2}\left||e^{-t}| - |e^{t}| \right|
\end{array} \]

\[ \everymath{\displaystyle}
\arraycolsep=1.8pt\def\arraystretch{2.5}
\begin{array}{rcl}
    |\sin(x+it)| & = & \left| \frac{e^{ix-t}-e^{-ix+t}}{2i} \right|\\
    & = & \left| \frac{e^{ix}e^{-t}-e^{-ix+t}e^{t}}{2i} \right|\\
    \mbox{($\triangle$-ineq)}& \geq & \frac{1}{2} \left| |e^{ix}||e^{-t}| - |e^{-ix}||e^{t}| \right|\\
    & = & \frac{1}{2}\left||e^{-t}| - |e^{t}| \right|
\end{array} \]

Note that the inequalities of both cases are satisfied for all $x\in\R$. Either $e^t$ or $e^{-t}$ converges to $\infty$ (and the other to 0), when $t\to \pm \infty$. Therefore, $\forall \varepsilon > 0$, there exists $N$ such that for every $x\in \R$:
\[ \frac{1}{\varepsilon} < \frac{1}{2}\left||e^{-t}| - |e^{t}| \right| \begin{array}{l}
    \leq |\cos(x+it)|\\[0.7 em]
    \leq |\sin(x+it)|
\end{array}\hspace{2em} \forall t > N. \]
So the limit is uniform on $x$.